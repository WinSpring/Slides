%\begin{frame}{TODO}
%Vorteile:
%\begin{itemize}
%\item Kultur
%\item Politik: politische Stabilität (wie Bayern mit CSU)
%\item Bevölkerung
%\item Religion
%\item Geschichte (Vietnamesen im Ausland nach dem Vietnamkrieg)
%\end{itemize}
%\end{frame}


\begin{frame}{Allgemeine Informationen}

\begin{columns}[T]% position: T top, b bottom or c centering
\column{0.5\textwidth}
\begin{tabular}{ll}
Hauptstadt & Hanoi \\
Währung	& Dong (VND) \\
Amtssprachen & Vietnamesisch \\
Bevölkerung & 101,8 Millionen (2025) \\
Landfläche & 331.690 km$^2$ \\
& \multirow{5}{6cm}{\includegraphics[height=10ex]{flagge_vn}} \\
&\\
Flagge&\\
&\\
&\\
\end{tabular}

\column{0.5\textwidth}
\vspace{-0.6cm}
\includegraphics[height=0.95\textheight]{bandovietnam1}
\scriptsize{https://piklab.vn/}

\end{columns}
\end{frame}


\begin{frame}{Geschichte Vietnams in Kürze (1)}
\setbeamercovered{transparent}

\begin{columns}[T]% position: T top, b bottom or c centering
\column{0.7\textwidth}
\vspace{-0.6cm}
\begin{itemize}
\item<1> Angeblich zwischen 2879 v. Chr. und 258 v. Chr.: Reich Văn Lang von Hồng-Bàng-Dynastie    
\item<1> 258 v. Chr. - 208 v. Chr. : Königreich Âu Lạc vom König An Dương Vương
\item<1> 208 v. Chr - 111 v. Chr. : Königreich Nam Việt von Triệu-Dynastie
\item<1> 111 v. Chr. - 938 n. Chr. : 1000-jährige chinesische Herrschaft in Vietnam
\item<2> 938 - 1858: Frühe Dynastien: Ngô, Đinh, Tiền Lê, Lý, Trần, Hồ, Mạc, Hậu Lê, Tây Sơn, Nguyễn
\item<2> 1858 - 1945: französische Kolonialherrschaft und Nguyễn-Dynastie
\item<2> 1945 - 1954: Unabhängigkeit und Indochinakrieg

\end{itemize}
\column{0.4\textwidth}
\vspace{-0.6cm}
\includegraphics[height=0.9\textheight]{bandovietnam}
\scriptsize{https://www.vntrip.vn/cam-nang/ban-do-vi-tri-dia-ly-viet-nam-42733}

\end{columns}

\end{frame}


\begin{frame}{Geschichte Vietnams in Kürze (2)}

\begin{columns}[T]% position: T top, b bottom or c centering
\column{0.7\textwidth}
%\vspace{-0.6cm}
\begin{itemize}
\item 1954 - 1975: Vietnamkrieg zwischen Nordvietnam und Südvietnam
 \begin{itemize}
 \item[+] Nordvietnam: Die Demokratische Republik Vietnam
 \item[+] Südvietnam: Die Republik Vietnam 
 \end{itemize}
\item seit 1975: Wiedervereinigung, Sozialistische Republik Vietnam 

\end{itemize}
\column{0.4\textwidth}
\vspace{-0.6cm}
\includegraphics[height=0.9\textheight]{bandovietnam}
\scriptsize{https://www.vntrip.vn/cam-nang/ban-do-vi-tri-dia-ly-viet-nam-42733}
\end{columns}
\end{frame}


\begin{frame}{Wirtschaft Vietnams in Kürze}
\setbeamercovered{transparent}
\begin{itemize}
\item<1> 1976 bis 1986: Zentralverwaltungswirtschaft (Planwirtschaft)
\item<1> Ab 1986: Wirtschaftserneuerung (Đổi mới)
\item<2> Vom ärmsten Land der Welt: 1990 lag das durchschnittliche Pro-Kopf-Einkommen bei ca. 100 Dollar pro Jahr
\item<2> zur Boomregion: in den vergangenen zehn Jahren betrug das Wirtschaftswachstum im Schnitt 6 - 7\%. Vietnam gehört damit zu den am schnellsten wachsenden Volkswirtschaften der Welt.
\item<2> Vietnam rangiert auf dem 34. Platz der wirtschaftlich stärksten Länder der Welt (Stand 2025)
\end{itemize}
\end{frame}


\begin{frame}{Wirtschaftserneuerung (Đổi mới)}
\setbeamercovered{transparent}
Ziel: Vietnam von einer zentral verwalteten Planwirtschaft in eine marktorientierte Wirtschaft umzuwandeln
\begin{itemize}
\item Wirtschaftliche Liberalisierung: Abkehr von der Planwirtschaft und Einführung von Marktkräften
\item Förderung des Privatsektors: Erlaubnis für private Unternehmen und Landwirte, ihre Erzeugnisse auf Märkten zu verkaufen
\item Öffnung für internationalen Handel und Investitionen: Integration Vietnams in die Weltwirtschaft
\item Bürokratieabbau und Verwaltungsreform
\item \ldots
\end{itemize}
\end{frame}


\begin{frame}{Bruttoinlandsprodukt (BIP) Vietnams in Milliarden US-Dollar}
\vspace{-0.4cm}
\centering
\includegraphics[height=0.95\textheight]{bip_vn}
\scriptsize{https://de.statista.com/statistik/daten/studie/324600/umfrage/bruttoinlandsprodukt-bip-von-vietnam/}
\end{frame}


\begin{frame}{Pro-Kopf-Bruttoinlandsprodukt in Vietnam im Laufe der Jahre (in US-Dollar)}
\vspace{-0.4cm}
\centering
\includegraphics[height=0.95\textheight]{bip_vn_prokopf}
\scriptsize{https://topi.vn/gdp-binh-quan-dau-nguoi-cua-viet-nam-qua-cac-nam.html und statista}
\end{frame}


\begin{frame}{Bruttowertschöpfung und Erwebstätige nach Sektoren}
%\vspace{-0.4cm}
\centering
\includegraphics[width=\textwidth]{bruttowertschoepfung_erwebstaetige}
\scriptsize{https://www.wko.at/statistik/laenderprofile/lp-vietnam.pdf}
\end{frame}


\begin{frame}{Export- und Importgüter 2023}
%\vspace{-0.4cm}
\centering
\includegraphics[width=\textwidth]{export_import_gueter_2023}
\scriptsize{https://www.wko.at/statistik/laenderprofile/lp-vietnam.pdf}
\end{frame}


\begin{frame}{Export- und Importländer 2024}
%\vspace{-0.4cm}
\centering
\includegraphics[width=\textwidth]{export_import_laender_2024}
\scriptsize{https://www.wko.at/statistik/laenderprofile/lp-vietnam.pdf}
\end{frame}


\begin{frame}{Tourimus}
\begin{columns}[T]% position: T top, b bottom or c centering
\column{0.6\textwidth}
\includegraphics[width=1.05\textwidth]{tourismus_diagramm}
\scriptsize{Internationale Ankünfte: 2013: $\textcolor{lmu@hyperlink}{\sim}$ 7,5 Millionen, 2019: $\textcolor{lmu@hyperlink}{\sim}$ 18 Millionen, 2024: $\textcolor{lmu@hyperlink}{\sim}$ 17,5 Millionen}

\scriptsize{(Quelle: https://thongke.tourism.vn/index.php/statistic/sub/6)}\\[2ex]

Tourismussektor machte in letzten Jahren ca. 6 - 8\% des BIP aus

\column{0.4\textwidth}
%\vspace{-0.6cm}
\includegraphics[width=\textwidth]{dulich_vn1}
\scriptsize{https://wttc.org/news/vietnams-travel-and-tourism-set-for-a-record-2024}
%\includegraphics[width=\textwidth]{dulichvn}
%\scriptsize{https://pystravel.vn/tin/6811-ban-do-du-lich-viet-nam.html}

\end{columns}
\end{frame}


\begin{frame}{Treiber des Wirtschaftswachstums Vietnam}
%https://aquis-capital.com/de/news/vietnam-economic-growth-dynamics-opportunities-and-outlook
\begin{itemize}
\item Junge und wachsende Bevölkerung ($\textcolor{lmu@hyperlink}{>}$ 50\% unter 35 Jahre alt) $\textcolor{lmu@hyperlink}{\Rightarrow}$ ein Vorteil für Arbeitsmarkt, Innovation und Konsum
\item Wachsende Mittelschicht ($\textcolor{lmu@hyperlink}{>}$ 50\% der Bevölkerung) $\textcolor{lmu@hyperlink}{\Rightarrow}$ treibt den Konsum und schafft neue Märkte
\item Ausländische Direktinvestitionen (FDI): Internationale Unternehmen wie Samsung, LG, Apple, Intel, Nike, Adidas, Bosch, Infineon haben Produktionsstätten nach Vietnam verlagert
\item Technologische Entwicklung: Vietnam entwickelt sich von einem Produktionsstandort zu einem Zentrum für IT-Dienstleistungen, Start-ups und digitale Innovation
\end{itemize}
\end{frame}


\begin{frame}{Herausforderungen und Risiken}
%https://aquis-capital.com/de/news/vietnam-economic-growth-dynamics-opportunities-and-outlook
\begin{itemize}
\item Globale Abhängigkeit: Vietnams Wirtschaft ist stark exportorientiert und damit anfällig für weltweite Konjunkturschwankungen
\item Regulatorische Hürden: Rechtliche Rahmenbedingungen entwickeln sich, sind aber nicht immer stabil vorhersehbar
\item Umwelt- und Infrastrukturfragen: Das schnelle Wachstum erfordert Investitionen in nachhaltige Lösungen
\end{itemize}
\end{frame}


\begin{frame}{Zusammenfassung}
Konsequente Reformen, internationale Vernetzung und eine junge, dynamische Bevölkerung sind die treibenden Kräfte des Wirtschaftswachstums Vietnams \\[2ex]
%Das Wirtschaftswachstum Vietnam ist ein Paradebeispiel dafür, wie konsequente Reformen, internationale Vernetzung und eine junge, dynamische Bevölkerung eine Volkswirtschaft verändern können\\[2ex]

Für Investoren, Unternehmen und politische Entscheidungsträger bleibt Vietnam ein Land mit enormem Potenzial – ein Markt, der heute schon eine wichtige Rolle spielt und in Zukunft noch mehr an globaler Bedeutung gewinnen wird\\[2ex]

(Quelle: https://aquis-capital.com/de/news/vietnam-economic-growth-dynamics-opportunities-and-outlook)
\end{frame}


